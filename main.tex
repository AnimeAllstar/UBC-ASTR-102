\documentclass[10pt,landscape]{article}
\usepackage{multicol}
\usepackage{calc}
\usepackage{ifthen}
\usepackage[landscape]{geometry}
\usepackage{hyperref}

% To make this come out properly in landscape mode, do one of the following
% 1.
%  pdflatex latexsheet.tex
%
% 2.
%  latex latexsheet.tex
%  dvips -P pdf  -t landscape latexsheet.dvi
%  ps2pdf latexsheet.ps


% If you're reading this, be prepared for confusion.  Making this was
% a learning experience for me, and it shows.  Much of the placement
% was hacked in; if you make it better, let me know...


% 2008-04
% Changed page margin code to use the geometry package. Also added code for
% conditional page margins, depending on paper size. Thanks to Uwe Ziegenhagen
% for the suggestions.

% 2006-08
% Made changes based on suggestions from Gene Cooperman. <gene at ccs.neu.edu>


% To Do:
% \listoffigures \listoftables
% \setcounter{secnumdepth}{0}


% This sets page margins to .5 inch if using letter paper, and to 1cm
% if using A4 paper. (This probably isn't strictly necessary.)
% If using another size paper, use default 1cm margins.
\ifthenelse{\lengthtest { \paperwidth = 11in}}
	{ \geometry{top=.5in,left=.5in,right=.5in,bottom=.5in} }
	{\ifthenelse{ \lengthtest{ \paperwidth = 297mm}}
		{\geometry{top=1cm,left=1cm,right=1cm,bottom=1cm} }
		{\geometry{top=1cm,left=1cm,right=1cm,bottom=1cm} }
	}

% Turn off header and footer
\pagestyle{empty}
 

% Redefine section commands to use less space
\makeatletter
\renewcommand{\section}{\@startsection{section}{1}{0mm}%
                                {-1ex plus -.5ex minus -.2ex}%
                                {0.5ex plus .2ex}%x
                                {\normalfont\large\bfseries}}
\renewcommand{\subsection}{\@startsection{subsection}{2}{0mm}%
                                {-1explus -.5ex minus -.2ex}%
                                {0.5ex plus .2ex}%
                                {\normalfont\normalsize\bfseries}}
\renewcommand{\subsubsection}{\@startsection{subsubsection}{3}{0mm}%
                                {-1ex plus -.5ex minus -.2ex}%
                                {1ex plus .2ex}%
                                {\normalfont\small\bfseries}}
\makeatother

% Define BibTeX command
\def\BibTeX{{\rm B\kern-.05em{\sc i\kern-.025em b}\kern-.08em
    T\kern-.1667em\lower.7ex\hbox{E}\kern-.125emX}}

% Don't print section numbers
\setcounter{secnumdepth}{0}


\setlength{\parindent}{0pt}
\setlength{\parskip}{0pt plus 0.5ex}


% -----------------------------------------------------------------------

\begin{document}

\raggedright
\footnotesize
\begin{multicols}{3}


% multicol parameters
% These lengths are set only within the two main columns
%\setlength{\columnseprule}{0.25pt}
\setlength{\premulticols}{1pt}
\setlength{\postmulticols}{1pt}
\setlength{\multicolsep}{1pt}
\setlength{\columnsep}{2pt}

\begin{center}
     \Large{\textbf{UBC ASTR 102 Formula Sheet}} \\
\end{center}

\section{Formulae}

$D = \frac{ad}{206,265}$ \\
$d$ = distance, $D$ = diameter,  $a$ = angular size (arcseconds) \\ 

$c = \lambda \nu$ \\ 
$\lambda$ = wavelegth, $\nu$ = frequency \\

$F = \sigma T^4$ \\
$F$ = energy flux (power per unit surface area), \\$T$ = temperature \\

$\lambda_{max}$ (in nm) $=$ $(3.0 \times 10^6$ K nm) / $T$ (in K) \\

$E = h\nu = \frac{hc}{\lambda}$ \\
E = energy \\

$\frac{1}{\lambda} = R \left ( \frac{1}{n^2} - \frac{1}{m^2}\right )$ \\
$n$ = final state, $m$ = initial state \\

$\frac{\Delta\lambda}{\lambda_0} = \frac{\lambda - \lambda_0}{\lambda_0} = \frac{v_r}{c}$\\
$\lambda_0 =$ original wavelength, $v_r =$ radial velocity\\
negative $v_r$ = source is approaching\\

$F = ma$ \\
$F$ = force (N), $m$ = mass (kg), $a$ = acceleration (ms$^{-2}$)\\

$F = \frac{Gm_1m_2}{r^2}$\\
$r$ = distance between $m_1$ and $m_2$\\

$E_k = \frac{1}{2}mv^2$\\
$E_k$ = kinetic energy\\

$E_p = mgh$\\
$E_p$ = potential energy\\

$U = -\frac{GMm}{r}$\\
$U$ = gravitational potential energy for object in orbit\\

$v_{esc} = \sqrt{\frac{2GM}{r}}$\\
$v_{esc}$ = escape velocity, $M$ = larger mass\\

$L=4\pi\sigma T^4R^2$\\
$L$ = Luminosity (Watts), $R$ = radius\\

$E = mc^2$

$d$ (parsecs) $= \frac{1}{p}$ (arcseconds)\\
$d$ = distance, $p$ = parallax angle\\

$v_t = 4.74 \mu d$\\
$v_t$ = transverse (tangential) velocity (kms$^{-1}$),\\ $\mu$ = proper motion (arcsec/year), $d$ = distance (parsec)\\

$v = \sqrt{v_t^2+ v_r^2}$\\

$b = \frac{L}{4\pi d^2}$\\
$b$ = apparent brightness (Wm$^{-2}$)

$\frac{L_1}{L_2} = \left( \frac{d_1}{d_2}\right)^2 = \frac{b_1}{b_2}$\\

$\frac{b_1}{b_2}$ (ratio of brightness of stars) $ = (2.512)^{\Delta m}$\\
$\Delta m = $ difference in apparent magnitude of stars\\

$m_2 - m_1 = 2.5 \log_{10} \left(\frac{b_1}{b_2}\right)$\\
$m_1, m_2 = $ apparent magnitudes, $b_1, b_2 = $ brightness\\

$m - M = 5 \log d - 5$\\
$m = $ apparent magnitude, $M = $ absolute magnitude\\
$d = $ distance in parsecs.\\

$\frac{b_V}{b_B},\frac{b_B}{b_U}$ = color ratios (lower is hotter)\\

$M_1 + M_2 = \frac{a^3}{P^2}$ (Kepler’s $3^{rd}$ law for binary stars)\\
$a$ = orbital separation (in AU), $P$ = orbital period (in years)\\

$\frac{M_1}{M_2} = \frac{a_2}{a_1}$\\

$\frac{L}{L_\odot} \simeq \left( \frac{M}{M_\odot} \right)^{3.5}$\\

$\tau_{MS} \simeq \left( \frac{M}{M_\odot} \right) \left( \frac{L_\odot}{L} \right)$ Gyr\\
$\tau_{MS} = $ main sequence lifetime, Gyr = Giga year\\

$E = fMc^2$\\
Energy derived from $H->He$ conversion\\
$f = 7 \times 10^{-4}$\\

$L = \frac{E}{t}$\\
$L$ is Luminosity, $t$ is time\\

$Lt = fMc^2$

$t = \frac{fMc^2}{L} \propto \frac{M}{L} \propto \frac{M}{M^{3.5}} \propto M^{-2.5}$

$r_s = \frac{2GM}{c^2}$\\
black hole's event horizon, schwarzschild radius\\

$P=  \frac{2\pi r}{v}$\\
The Sun’s orbital period\\

$M = \frac{rv^2}{G}$\\
amount of matter inside the Sun’s orbit\\
\section{Constants}

\begin{tabular}{@{}ll@{}}
$c$ (speed of light)    &   $3 \times 10^8$ ms$^{-1}$ \\
$\sigma$ (Stefan-Boltzmann constant)   &    $5.67 \times 10^{-8}$ Wm$^{-2}$K$^{-4}$ \\
$h$ (Planck’s constant) &   $6.626 \times 10^{-34}$ Js\\
$R$ (Rydberg constant)  &   $1.097 \times 10^{7}$ m$^{-1}$\\
$G$ (Newton’s constant)  &   $6.67 \times 10^{-11}$ Nm$^{2}$kg$^{-2}$\\
$g$ (gravity)   &   $9.8$ ms$^{-2}$
\end{tabular}

\section{Conversions}

\begin{tabular}{@{}ll@{}}
\verb!1 degree!    & 60 arcminutes \\
\verb!1 arcminute! & 60 arcseconds \\
\verb!1 lightyear! & $9.46 \times 10^{12}$ km\\
\verb!1 AU! & $1.5 \times 10^{11}$ m\\
\verb!1 parsec! & $3.26$ lightyears\\
\verb!1 m! & $10^9$ nm\\
\verb!1 Gyr! & $10^9$ years\\
\end{tabular}

\section{Chemical Processes}

proton proton chain\\
triple alpha process\\

\section{Diagrams}

Blackbody spectrum\\
Cumulative Luminosity\\
Temperature and Density from the core\\
HR diagram\\

\section{Miscellaneous 1}
\begin{itemize}
    \item absolute magnitude: apparent magnitude a star would have if it were located exactly 10 parsecs from Earth
    \item 10\% of the mass of the star will be converted from H to He while it is on the main sequence.
    \item emission nebulae glow because UV knocking off electrons, which then recombine, and release photons in the visible spectrum.
    \item dark nebulae is caused due to dust, required for star formation, don't block infrared.
    \item reflection nebula contain fine grains of dust, they scatter short wavelength blue light efficiently, causing a reddening effect on stars.
    \item Protostars are initial stars formed in dark nebulae. 
    \item Larger protostars take a shorter amount of time to reach the main sequence.
    \item planet $< 0.012 M_\odot <$ brown dwarf $< 0.08 M_\odot <$ main sequence stars $< 200 M_\odot$
    \item high mass stars burn through their fuel first
    \item open clusters - held together by gravitational forces
    \item stellar association - fast moving cluster, mostly O, B stars (OB association)
    \item globular clusters - large, old clusters
    \item CO helps predict star formation
    \item A red dwarf is a main-sequence star with less than about 0.4 solar masses. Helium star.
    \item End of H fusion: core starts to cool, pressure in the core starts to decrease, the core compresses/shrinks, the temperature to increase again, and outer layers will expand.
    \item In red giants with a mass less than about 2 to 3 $M\odot$ but still more than 0.4 $M\odot$, helium fusion begins explosively and suddenly, in what is called the helium flash. 
    \item If the core is contracting (no fusion): Core temperature goes up, Heat transfers to the outer layers, The outer layers expand, (Low mass: Luminosity increases)
    \item If the core is expanding (fusion): Core temperature goes down, Less heat to the outer layers, The outer layers shrink (Low mass: Luminosity decreases)
    \item black hole properties; mass, charge and spin.
\end{itemize}
\section{Miscellaneous 2}

\begin{tabular}{@{}ll@{}}
\verb!violet light!    & 400 $nm$ \\
\verb!red light!    & 700 $nm$ \\
\verb!blue shift!    & approaching (short $\lambda$) \\
\verb!red shift!    & receding (longer $\lambda$) \\
$L_\odot$    &  Luminosity of the sun\\
$M_\odot$    &  Mass of the sun\\
\verb!layers of sun!   &   Thermonuclear $->$ radiative $->$ convection\\
$m - M$ & Distance modulus (apparent - absolute)\\
OBAFGKM & current ordering of stars (hot to cold)\\
L, T, Y & brown dwarf spectral classes
\end{tabular}

\clearpage

\section{Miscellaneous (post midterm)}
\begin{itemize}
    \item Population II stars are old and metal poor
    \item Population I stars are young and metal heavy
    \item Cepheid variables brighten and fade because their outer envelope cyclically expands and contracts. They are found throughout the galaxy. Pulsation period 1 to 50 days. Luminosity positively related to period.
    \item RR Lyrae stars. Found in globular clusters. Pulsation periods less than a day. All have the same luminosity.
    \item They are very bright and can be seen to large distances, and obey a relationship between luminosity and pulsation period
    \item protostar $->$ H is over $->$ red giant $->$ helium flash $->$ horizontal branch $->$ helium is used up $->$ asymptotic giant branch $->$ thermal pulses $->$ planetary nebula $->$ end of nuclear reactions $->$ white dwarf
    \item photodisintegration happens in core collapse of core mass $>$ 8 $M_\odot ->$ star becomes a neutron star.
    \item type Ia supernova: red giant companion, white dwarf explodes. (No He, H. Produces Si)
    \item When the total mass of the white dwarf approaches the
Chandrasekhar limit (1.4 $M_\odot ->$). “standard candle”
    \item type Ib supernova: Core collapse of massive red giant. outer layers have no H(No H. Produces He)
    \item type Ib supernova: Core collapse of massive red giant. outer layers have no H or He (No H or He)
    \item type II supernova: Core collapse of massive red giant. outer layers intact. (Produces H)
    \item $M > 25 M_\odot ->$ black hole
    \item $M > 8 M_\odot ->$ neutron star
    \item $M > 0.4 M_\odot ->$ white dwarf
    \item $M > 0.08 M_\odot ->$ red dwarf star
    \item $M < 0.08 M_\odot ->$ brown dwarf
    \item gravitational waves: Ripples in the fabric of spacetime generated by the acceleration of matter
    \item short-duration gamma-ray bursts: neutron star mergers
    \item long-distance gamma-ray bursts: Type Ic supernova
    \item We can observe where the neutral hydrogen atoms are in our Galaxy and in others by looking for emission corresponding to the spin flip of the electron, from aligned with the proton’s spin to aligned. The spin flip results in the release of a photon of wavelength 21 cm (frequency 1420 MHz)
    \item MACHOs: brown dwarfs, white dwarfs, or black holes, are called massive compact halo objects
    \item WIMPs: Weakly Interacting Massive Particles
    \item Elliptical galaxies have virtually no interstellar gas or dust, and no evidence of young stars in most elliptical galaxies. For the most part, star formation in elliptical galaxies ended long ago.
\end{itemize}

\end{multicols}
\end{document}
